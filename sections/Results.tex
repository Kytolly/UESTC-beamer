\begin{frame}{已完成的研究工作及成果}
    \begin{block}{已完成的研究工作简介}
      \begin{itemize}
        \setlength{\itemsep}{6pt}
        \item XXXX
        \item XXXX
        \item XXXX
        \item XXXX
        \item XXXX
        \item XXXX
      \end{itemize}
    \end{block}
\end{frame}

\subsection{研究工作一}
\begin{frame}{研究工作一}
  \begin{block}{无编号公式}
    无编号公式示例:
    $$
      k:[-\pi,\pi] \rightarrow [0,1]
    $$
  \end{block}
\end{frame}

\subsection{研究工作二}

\begin{frame}{研究工作二}
  \begin{block}{有编号公式}
    \begin{itemize}
      \item 有编号公式示例:输入为图像 
      \begin{equation}
        \vx\in\reals^{C_{\text{in}}\times H\times W}
      \end{equation}
      其中 $C_{\text{in}}$ 表示通道, $H$ 表示图像高度, $W$ 表示图像深度.
    \end{itemize}
  \end{block}
\end{frame}

\subsection{研究工作三}

\begin{frame}{研究工作三}
  \begin{block}{表格}
    表格示例, 如表 \ref{tab:unique_values} 所示. 

    \begin{table}[htbp]
      \small
      \centering
      \caption{train.csv 每列非重复元素个数}
      \label{tab:unique_values}
      \begin{tabular}{lc}
        \toprule
        column & \# unique values \\
        \midrule
        posting\_id & 34250 \\
        image & 32412 \\
        image\_phash & 28735 \\
        title & 33117 \\
        label\_group & 11014 \\
        \bottomrule
      \end{tabular}
    \end{table}
  \end{block}
\end{frame}

\subsection{研究工作四}

\begin{frame}{研究工作四}
  \begin{block}{并排图片}
    并排图片示例.
  \end{block}
  \begin{figure}[htbp]
    \centering
    \begin{minipage}[t]{0.48\textwidth}
      \centering
      \includegraphics[width=3cm]{uestclogo}
      \caption{并排图片1}
      \label{fig:left_side}
    \end{minipage}
    \begin{minipage}[t]{0.48\textwidth}
      \centering
      \includegraphics[width=3cm]{uestclogo}
      \caption{并排图片2}
      \label{fig:right_side}
    \end{minipage}
  \end{figure}
\end{frame}